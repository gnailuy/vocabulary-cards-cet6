% ========== 01
\begin{flashcard}[NOTHING SEEK, NOTHING FIND]{
VIEW

view
}
\vspace*{\stretch{1}}
\begin{center}
VIEW n. vt. \textipa{[vju\textlengthmark]}
\end{center}
n. 1. 看法,见解,观点

He was sent to jail for his political \textit{views}.

n. 2. 观察,眼界;景色,风景

The car turned the corner and was lost to our \textit{view}.

We walked up peaks to see \textit{views}.

vt. 1. 看待,考虑,估量;观察,看

It seems probable that he will \textit{view} our request with favour.

Several possible buyers have come to \textit{view} the house.

\vspace*{\stretch{1}}
\end{flashcard}
% ========== 02
\begin{flashcard}[NOTHING SEEK, NOTHING FIND]{
VARY

vary
}
\vspace*{\stretch{1}}
\begin{center}
VARY vi. vt. \textipa{[\textprimstress v\textepsilon \textschwa ri]}
\end{center}
vi. 1. 变化,有不同,呈差异

Her mood \textit{varies} with the weather.

vt. 1. 改变,使不同

Old people don't like to \textit{vary} their habits.

\vspace*{\stretch{1}}
\end{flashcard}
% ========== 03
\begin{flashcard}[NOTHING SEEK, NOTHING FIND]{
MINORITY

minority
}
\vspace*{\stretch{1}}
\begin{center}
MINORITY n. \textipa{[mai\textprimstress n\textopeno riti]}
\end{center}
1. 少数,少数派

Such people are in the \textit{minority}.

2. 少数民族

The Chinese people include over 50 national \textit{minorities} besides the Hans.

\vspace*{\stretch{1}}
\end{flashcard}
% ========== 04
\begin{flashcard}[NOTHING SEEK, NOTHING FIND]{
IMAGE

image
}
\vspace*{\stretch{1}}
\begin{center}
IMAGE n. \textipa{[\textprimstress imid\textyogh]}
\end{center}
1. 形象

A criminal charge is not good for a politician's \textit{image}.

\vspace*{\stretch{1}}
\end{flashcard}
% ========== 05
\begin{flashcard}[NOTHING SEEK, NOTHING FIND]{
CONCEPT

concept
}
\vspace*{\stretch{1}}
\begin{center}
CONCEPT n. \textipa{[\textprimstress k\textopeno nsept]}
\end{center}
1. 概念;观念,思想

He speaks in \textit{concepts} rather than specifics.

\vspace*{\stretch{1}}
\end{flashcard}
% ========== 06
\begin{flashcard}[NOTHING SEEK, NOTHING FIND]{
ASSUME

assume
}
\vspace*{\stretch{1}}
\begin{center}
ASSUME vt. \textipa{[\textschwa \textprimstress su\textlengthmark m]}
\end{center}
1. 假定,假设,臆断

He \textit{assumed} the report (to be) valid.

2. 承担,担任,就职

He \textit{assumed} the duty.

3. 呈现,具有

The situation \textit{assumed} a threatening character.

\vspace*{\stretch{1}}
\end{flashcard}
% ========== 07
\begin{flashcard}[NOTHING SEEK, NOTHING FIND]{
OBSCURE

obscure
}
\vspace*{\stretch{1}}
\begin{center}
OBSCURE a. vt. \textipa{[\textschwa b\textprimstress skju\textschwa]}
\end{center}
a. 1. 不出名的,不重要的

You may not have heard of the painter of this portrait---he is rather \textit{obscure}.

a. 2. 费解的;模糊不清的

The meaning of this poem is \textit{obscure}.

The room is too \textit{obscure} for reading.

vt. 1. 使难解,使变模糊

Words and sentences that \textit{obscure} the meaning must be discarded.

\vspace*{\stretch{1}}
\end{flashcard}
% ========== 08
\begin{flashcard}[NOTHING SEEK, NOTHING FIND]{
PARTICULARLY

particularly
}
\vspace*{\stretch{1}}
\begin{center}
PARTICULARLY ad. \textipa{[p\textschwa \textprimstress tikjul\textschwa li]}
\end{center}
1. 特别,尤其

Today is a \textit{particularly} fine day.

\vspace*{\stretch{1}}
\end{flashcard}
% ========== 09
\begin{flashcard}[NOTHING SEEK, NOTHING FIND]{
SUPPRESS

suppress
}
\vspace*{\stretch{1}}
\begin{center}
SUPPRESS vt. \textipa{[s\textschwa \textprimstress pres]}
\end{center}
1. 压制,镇压

All religious activities were \textit{suppressed}.

2. 禁止发表,查禁

\textit{suppress} a news story (news papers)

3. 抑制(感情等),忍住

He \textit{suppressed} his anger.

4. 阻止...的生长(或发展)

She was struggling to \textit{suppress} her sobs.

\vspace*{\stretch{1}}
\end{flashcard}
% ========== 10
\begin{flashcard}[NOTHING SEEK, NOTHING FIND]{
INTEGRITY

integrity
}
\vspace*{\stretch{1}}
\begin{center}
INTEGRITY n. \textipa{[in\textprimstress tegr\textschwa ti]}
\end{center}
1. 正直,诚实,诚恳

He is a man of \textit{integrity}.

\vspace*{\stretch{1}}
\end{flashcard}
